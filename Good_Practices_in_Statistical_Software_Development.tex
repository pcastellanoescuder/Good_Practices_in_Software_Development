\PassOptionsToPackage{unicode=true}{hyperref} % options for packages loaded elsewhere
\PassOptionsToPackage{hyphens}{url}
%
\documentclass[ignorenonframetext,]{beamer}
\setbeamertemplate{caption}[numbered]
\setbeamertemplate{caption label separator}{: }
\setbeamercolor{caption name}{fg=normal text.fg}
\beamertemplatenavigationsymbolsempty
\usepackage{lmodern}
\usepackage[utf8]{inputenc}
\usepackage{amssymb,amsmath}
\usepackage{ifxetex,ifluatex}
\usepackage{fixltx2e} % provides \textsubscript
\ifnum 0\ifxetex 1\fi\ifluatex 1\fi=0 % if pdftex
  \usepackage[T1]{fontenc}
  \usepackage[utf8]{inputenc}
  \usepackage{textcomp} % provides euro and other symbols
\else % if luatex or xelatex
  \usepackage{unicode-math}
  \defaultfontfeatures{Ligatures=TeX,Scale=MatchLowercase}
\fi
\usetheme[]{Copenhagen}
\usecolortheme{dolphin}
\usefonttheme{structurebold}
% use upquote if available, for straight quotes in verbatim environments
\IfFileExists{upquote.sty}{\usepackage{upquote}}{}
% use microtype if available
\IfFileExists{microtype.sty}{%
\usepackage[]{microtype}
\UseMicrotypeSet[protrusion]{basicmath} % disable protrusion for tt fonts
}{}
\IfFileExists{parskip.sty}{%
\usepackage{parskip}
}{% else
\setlength{\parindent}{0pt}
\setlength{\parskip}{6pt plus 2pt minus 1pt}
}
\usepackage{hyperref}
\hypersetup{
            pdfauthor={Alex Sánchez-Pla and Pol Castellano-Escuder},
            pdfborder={0 0 0},
            breaklinks=true}
\urlstyle{same}  % don't use monospace font for urls
\newif\ifbibliography
% Prevent slide breaks in the middle of a paragraph:
\widowpenalties 1 10000
\raggedbottom
\setbeamertemplate{part page}{
\centering
\begin{beamercolorbox}[sep=16pt,center]{part title}
  \usebeamerfont{part title}\insertpart\par
\end{beamercolorbox}
}
\setbeamertemplate{section page}{
\centering
\begin{beamercolorbox}[sep=12pt,center]{part title}
  \usebeamerfont{section title}\insertsection\par
\end{beamercolorbox}
}
\setbeamertemplate{subsection page}{
\centering
\begin{beamercolorbox}[sep=8pt,center]{part title}
  \usebeamerfont{subsection title}\insertsubsection\par
\end{beamercolorbox}
}
\AtBeginPart{
  \frame{\partpage}
}
\AtBeginSection{
  \ifbibliography
  \else
    \frame{\sectionpage}
  \fi
}
\AtBeginSubsection{
  \frame{\subsectionpage}
}
\setlength{\emergencystretch}{3em}  % prevent overfull lines
\providecommand{\tightlist}{%
  \setlength{\itemsep}{0pt}\setlength{\parskip}{0pt}}
\setcounter{secnumdepth}{0}

% set default figure placement to htbp
\makeatletter
\def\fps@figure{htbp}
\makeatother

\def\begincols{\begin{columns}}
\def\begincol{\begin{column}}
\def\endcol{\end{column}}
\def\endcols{\end{columns}}

\title{Good Practices in (Statistical)\\
Software Development}
\author{Alex Sánchez-Pla and Pol Castellano-Escuder}

\date{Genetics, Microbiology and Statistic Department. UB\\
Statistics and Bioinformatics Unit. VHIR\\
Grup de Recerca en Bioestadí­stica i Bioinformatica (GRBio)\\

\vspace{0.5cm}

\footnotesize{Jan 26, 2019 - BIOSTATNET Meeting}

\vspace{0.2cm}

\footnotesize{\textbf{\url{http://github.com/pcastellanoescuder/Good_Practices_in_Software_Development}}}
}

\begin{document}
\frame{\titlepage}

\begin{frame}

\begin{block}{Readme}

\begin{itemize}
\item
  License: Creative Commons Attribution-NonCommercial-ShareAlike 4.0
  International License
  \url{http://creativecommons.org/licenses/by-nc-sa/4.0/}
\item
  You are free to:

  \begin{itemize}
  \tightlist
  \item
    \textbf{Share} : copy and redistribute the material
  \item
    \textbf{Adapt} : rebuild and transform the material
  \end{itemize}
\item
  Under the following conditions:

  \begin{itemize}
  \tightlist
  \item
    \textbf{Attribution} : You must give appropriate credit, provide a
    link to the license, and indicate if changes were made.
  \item
    \textbf{NonCommercial} : You may not use this work for commercial
    purposes.
  \item
    \textbf{Share Alike} : If you remix, transform, or build upon this
    work, you must distribute your contributions under the same license
    to this one.
  \end{itemize}
\end{itemize}

\end{block}

\end{frame}

\begin{frame}{%
\protect\hypertarget{outline}{%
Outline}}

\begin{itemize}
\tightlist
\item
  Introduction to the session.
\item
  Statistics and Programming

  \begin{itemize}
  \tightlist
  \item
    The role of coding in statistical work
  \item
    Teaching programming to statisticians and data scientists
  \end{itemize}
\item
  Can we build better software (software better?)

  \begin{itemize}
  \tightlist
  \item
    When do we call a piece of software “good”?
  \item
    Which aspects should we account for
  \end{itemize}
\item
  Good practices for statistical software development
\end{itemize}

\end{frame}

\begin{frame}{%
\protect\hypertarget{introduction}{%
Introduction}}

\begin{itemize}
\tightlist
\item
  Presenter: Alex Sánchez Pla
\item
  Job 1: Associate Professor. Statistics department, UB. “Traditional
  approach”

  \begin{itemize}
  \tightlist
  \item
    Teach statistics (SI, ED), make research (GRBio)
  \item
    Mostly programmed for myself. Some R packages.
  \end{itemize}
\item
  Job 2: Head of Statistics and Bioinformatics Unit. “Real world”

  \begin{itemize}
  \tightlist
  \item
    Coordinate (more than do) applied biostatistics and bioinformatics.
  \item
    Lots of data, lots of programming
  \item
    Discovered new concepts: Validation, Regulation, SOPs, Best
    Practices, \ldots{}
  \end{itemize}
\end{itemize}

\end{frame}

\begin{frame}{%
\protect\hypertarget{what-is-software-development}{%
What is software development}}

\begin{itemize}
\item
  \emph{Software development is a process by which standalone or
  individual software is created using a specific programming language.
  It involves writing a series of interrelated programming code, which
  provides the functionality of the developed software}
\item
  Simple, yet functional. It embraces most programming activities we may
  be involved in,

  \begin{itemize}
  \tightlist
  \item
    From writing simple scripts, automating functions, running
    simulations.
  \item
    Creating web programs, mobile apps, etc.
  \item
    But also writing code for performing any type of statisytical
    analyses.
  \end{itemize}
\end{itemize}

\end{frame}

\begin{frame}{%
\protect\hypertarget{how-do-we-do-software-development}{%
How do we do software development?}}

\begin{figure}
\includegraphics[width=0.95\textwidth]{"images/programmingSurvey"}
\end{figure}

\end{frame}

\begin{frame}{%
	\protect\hypertarget{results1}{%
		Survey answers (1)}}

\begin{figure}
	\includegraphics[width=1\textwidth]{"images/results1"}
\end{figure}

\end{frame}

\begin{frame}{%
	\protect\hypertarget{results2}{%
		Survey answers (2)}}

\begin{figure}
	\includegraphics[width=0.7\textwidth]{"images/results2"}
\end{figure}

\end{frame}

\begin{frame}{%
	\protect\hypertarget{results3}{%
		Survey answers (3)}}

\begin{figure}
	\includegraphics[width=0.7\textwidth]{"images/results3"}
\end{figure}

\end{frame}

\begin{frame}{%
	\protect\hypertarget{results4}{%
		Survey answers (4)}}

\begin{figure}
	\includegraphics[width=0.7\textwidth]{"images/results4"}
\end{figure}

\end{frame}

\begin{frame}{%
\protect\hypertarget{how-important-is-programming-in-a-mathematicians-training}{%
How important is programming in a mathematician’s training?}}

\begin{figure}
\includegraphics[width=0.75\textwidth]{"images/coursesMathematics"}
\caption{Compulsory courses. Mathematics degree. UPC. (One course)}
\end{figure}

\end{frame}

\begin{frame}{%
\protect\hypertarget{how-important-is-programming-in-a-statisticians-training}{%
How important is programming in a statistician’s training?}}

\begin{figure}
\includegraphics[width=0.85\textwidth]{"images/coursesAppliedStatsUAB"}
\caption{Compulsory courses. Applied Statistics degree at UAB. (Three courses)}
\end{figure}

\end{frame}

\begin{frame}{%
\protect\hypertarget{how-important-is-programming-in-a-data-scientists-training}{%
How important is programming in a data scientist’s training?}}

\begin{figure}
\includegraphics[width=0.75\textwidth]{"images/coursesDataEngineeringUPC"}
\caption{Compulsory courses. Data Engineering degree. UPC. (Five courses)}
\end{figure}

\end{frame}

\begin{frame}{%
\protect\hypertarget{good-practices-in-statistical-software-development}{%
Good practices in (statistical) software development}}

\begin{itemize}
\tightlist
\item
  \emph{A good practice is not only a practice that is good, but a
  practice that has been proven to work well and produce good results,
  and is therefore recommended as a model}.
\item
  There is no universally accepted list of ``good practices’’ for
  software development but we can easily agree (can we?) on some
  practices that (maybe) can help our programs to be better.
\item
 When we talk about software, we must bear in mind that there are many different types of software and that, sometimes, "good practices" can be different between software types. Some types of software could be: CLI (Command Line Interface), GUI (Graphical User Interface), Scripts, Packages or Web Apps.
 
\end{itemize}

\end{frame}

\begin{frame}{%
\protect\hypertarget{best-practices-in-software-engineering}{%
Best practices in Software Engineering}}

\includegraphics[width=0.75\textwidth]{"images/bestPracticesSoftwareEngineering"}

\begin{itemize}
\tightlist
\item
  We don’t discuss this!
\end{itemize}

\end{frame}

\begin{frame}{%
\protect\hypertarget{good-practices-in-statistical-software-development-1}{%
Good practices in Statistical Software Development}}

\begin{itemize}
\tightlist
\item
  Instead I have browsed through some materials which focus on
  statistical software development
\end{itemize}

\begin{center}
\includegraphics[width=0.9\textwidth]{"images/oodPractices4RProgramming"}
\end{center}

\end{frame}

\begin{frame}{%
\protect\hypertarget{gp-document-your-code}{%
GP: Document your code}}

\begin{itemize}
\tightlist
\item
  \textbf{Comment} as much as you can.
\item
  Write \textbf{help} for your programs.
\item
  Create \textbf{user manuals} and \textbf{user guides}.
\item
  Use tools to facilitate documentation building.

  \begin{center}
  \includegraphics[width=0.7\textwidth]{"images/roxygen"}
  \end{center}
\end{itemize}

\end{frame}

\begin{frame}{%
\protect\hypertarget{gp-modularize}{%
GP: Modularize}}

”Modular programming consists in separate the functionality of a program
into independent and interchangeable modules.”

\begin{itemize}
\tightlist
\item
  If a piece of code has to be used repeteadly or it is too long to
  affect code legibility it is recommended to encapsulate it into some
  type of \emph{module}
\item
  Modularization can happen at distinct levels.

  \begin{itemize}
  \tightlist
  \item
    \emph{functions}
  \item
    \emph{classes and methods} (OOP)
  \item
    \emph{libraries} or \emph{packages}
  \end{itemize}
\end{itemize}

\begin{center}
\includegraphics[width=0.6\textwidth]{"images/modularProgramming"}
\end{center}

\end{frame}

\begin{frame}{%
\protect\hypertarget{gp-literate-programming-enhances-reproducibility}{%
GP: Literate programming enhances reproducibility}}

\begin{itemize}
\item
  \textbf{Literate programming}: Enhances traditional software
  development by embedding code in explanatory essays and encourages
  \emph{treating the act of development as one of communication}.
\item
  \textbf{Reproducible research}: Embeds executable code in research
  reports and publications, with the aim of \emph{allowing readers to
  re-run the analyses described}.
\end{itemize}

\begin{center}
\includegraphics[width=0.6\textwidth]{"images/orgMode"}
\end{center}

\end{frame}

\begin{frame}{%
\protect\hypertarget{reproducible-research-with-r-and-rstudio}{%
Reproducible research with R and Rstudio}}

\begin{itemize}
\tightlist
\item
  Using literate programing for reproducible research is a change of
  paradigm.
\item
  Coding, Analyzing and Reporting become a unique piece of work.
\end{itemize}

\begin{center}
\includegraphics[width=0.6\textwidth]{"images/reproducibleResearchWithR"}
\end{center}

\end{frame}

\begin{frame}{%
\protect\hypertarget{gp-use-control-version-systems-cvs}{%
GP: Use Control Version Systems (CVS)}}

\begin{center}
\includegraphics[width=0.8\textwidth]{"images/whyWeNeedCVS"}
\end{center}

\begin{itemize}
\tightlist
\item
  \textbf{Version control systems} are a category of software tools that
  help a software developper/team manage changes to source code over
  time.
\item
  \textbf{Version control software} keeps track of every modification to
  the code in a special kind of database.

  \begin{itemize}
  \tightlist
  \item
    If a mistake is made, developers can turn back the clock and compare
    earlier versions of the code to help fix the mistake while
    minimizing disruption to all team members.
  \end{itemize}
\end{itemize}

\end{frame}

\begin{frame}{%
\protect\hypertarget{type-of-control-version-systems}{%
Type of Control version systems}}

\begin{center}
\includegraphics[width=0.8\textwidth]{"images/ccvs-dcvs"}
\end{center}

\end{frame}

\begin{frame}[fragile]{%
\protect\hypertarget{git-and-github-a-standard-de-facto}{%
Git and Github: a standard \emph{de facto}}}

\begin{itemize}
\tightlist
\item
  \texttt{GitHub} is a hosting service for \texttt{Git} repositories.
\item
  \texttt{GitHub} offers all the \texttt{Git} functionality adding a
  series of own characteristics.

  \begin{itemize}
  \tightlist
  \item
    Web and Desktop-based graphical interface.
  \item
    \emph{Access control}, \emph{bug tracking}, \emph{task management},
    \emph{wikis} \ldots{}
  \end{itemize}
\item
  \emph{GitHub brings together the worlds largest community of
  developers to \textbf{discover}, \textbf{share}, and \textbf{build}
  better software}
\end{itemize}

\begin{center}
\includegraphics[width=0.7\textwidth]{"images/gitAndGithub"}
\end{center}

\end{frame}

\begin{frame}{%
\protect\hypertarget{gp-share-your-programs}{%
GP: Share your programs}}

\begin{itemize}
\tightlist
\item
  Sharing software is one great way to improve it.

  \begin{itemize}
  \tightlist
  \item
    Developers take care of details so that users will be staisfied with
    the program.
  \item
    Users can provide feedback and can help improve the program.
  \end{itemize}
\item
  Sharing can be done in many different ways

  \begin{itemize}
  \tightlist
  \item
    Put a library in a repository
  \item
    Create a web page
  \item
    Sell the program
  \end{itemize}
\item
  Many scientist’s work is better known by their software tools than by
  their papers
\end{itemize}

\end{frame}

\begin{frame}{%
\protect\hypertarget{gp-verify-and-validate-your-software}{%
GP: Verify and Validate\footnote{\url{https://www.softwaretestinghelp.com/difference-between-verification-vs-validation/}}}  your software}

\begin{itemize}
\tightlist
\item
  How do we know that the code does what it is intended to do?

  \begin{itemize}
  \tightlist
  \item
    Prepare a battery of tests, apply them. Document it.
  \end{itemize}
\item
  How do we know that our code contains no mistakes?

  \begin{itemize}
  \tightlist
  \item
    Make someone else review your work. Document it.
  \end{itemize}
\item
  How do we know if the code can be installed and set to work?

  \begin{itemize}
  \tightlist
  \item
    Make someone else install and run the programs. Document it.
  \end{itemize}
\end{itemize}

\begin{center}
\includegraphics[height=0.4\textheight]{"images/Verification-and-Validation"}
\end{center}


\end{frame}

\begin{frame}{%
\protect\hypertarget{gp-optimize-efficiently}{%
GP: Optimize efficiently}}

\begin{itemize}
\tightlist
\item
  Optimization may improve a program’s performance \emph{if it doesn’t
  consume more time than the time it saves}.

  \begin{itemize}
  \tightlist
  \item
    Before you start to optimise your code, ensure you know where the
    bottleneck lies; use a code profiler.
  \item
    Use the appropriate data structures (e.g.~matrices instead of data
    frames) and
  \item
    Use the right functions to work with them (Use specialised row and
    column functions whenever possible)
  \end{itemize}
\item
  Learn to use techniques that can improve speed orders of magnitude.
  Apply if needed.

  \begin{itemize}
  \tightlist
  \item
    Parallelization can be great if many processes will be executed “in
    parallel” (e.g.~Monte Carlo SImulations)
  \item
    Learn to combine languages. Consider re-writing key parts of your
    code in C++.
  \end{itemize}
\end{itemize}

\end{frame}

\begin{frame}{%
\protect\hypertarget{gp-manage-error-control-and-detection-proactively}{%
GP: Manage error control and detection proactively}}

\begin{itemize}
\tightlist
\item
  Testing and reviewing can detect some errors

  \begin{itemize}
  \tightlist
  \item
    Some of these (e.g.~programmers’ errors) can be removed.
  \item
    Others (e.g.~wrong input) cannot be avoided.
  \end{itemize}
\item
  Learn \textbf{how to debug} your code
\item
  \textbf{Plan reaction to errors} according to their severity.
\item
  Learn how to do \textbf{Exception handling} tho help avoid program
  crashes.
\end{itemize}

\begin{center}
\includegraphics[height=0.3\textheight]{"images/errorHandling"}
\end{center}

\end{frame}

\begin{frame}{%
\protect\hypertarget{gp-plan-the-process}{%
GP: Plan the process}}

\textbf{Think, plan, code it!} And NOT in another order!\\ 
This will not take much time but, however, will save us a significant
amount of time.

\begin{itemize}
\tightlist
\item
  Software development is a \textbf{process}
\item
  There exist a myriad of methodologies\footnote{\url{https://www.synopsys.com/blogs/software-security/top-4-software-development-methodologies/}} and tools for planning and
  executing. Use the one you prefer.
\end{itemize}

\begin{center}
\includegraphics[width=0.9\textwidth]{"images/softwareDevelopmentProcess"}
\end{center}

\begin{itemize}
	\tightlist
	\item
	What do we want? How are we going to do it? 
\end{itemize}

\end{frame}

\begin{frame}{%
		\protect\hypertarget{gp-design}{%
			GP: Design}}
	\begin{itemize}
		\item
		Design or aesthetics may not seem so important, but nevertheless they are,
		especially in web applications.
		\item
		Sometimes, a very good software with a design that does not attract the
		user may not give the maximum performance that is expected of it.
		However, if the software is friendly and is accompanied by a good design,
		we can obtain better results.
		\item
		To achieve this goal, we suggest the use of html, css, markdown,
		rmarkdown, javascript, latex...
    \end{itemize}
    
\end{frame}

\begin{frame}{%
\protect\hypertarget{in-summary}{%
Conclusions}}

\begin{itemize}
\item
  Software development may be a “secondary” activity or may be our main
  source of work.
\item
  The more we know about software development the better we can do the
  job.
\item
  Applying a few good practices can considerably improve our
  performance.
\item
  Remember that:
\end{itemize}

\emph{"You shouldn’t feel ashamed about your code -if it solves the
problem, it’s perfect the way it is. But also, it could always be
better"}

-Hadley Wickam

\end{frame}

\begin{frame}{%
		\protect\hypertarget{summary}{%
			Good Practices Summary}}
	
	\begin{itemize}
		\item Documentation
		
		\item Modularize
		
		\item Literate programming
		
		\item Version Control
		
		\item Share your programs
		
		\item Verify and Validate
		
		\item Optimize efficiently
		
		\item Error Control
		
		\item Planification
		
		\item Design
		
	\end{itemize}
	
\end{frame}
	
\end{document}
